% ****** Start of file apssamp.tex ******
%
%   This file is part of the APS files in the REVTeX 4.2 distribution.
%   Version 4.2a of REVTeX, December 2014
%
%   Copyright (c) 2014 The American Physical Society.
%
%   See the REVTeX 4 README file for restrictions and more information.
%
% TeX'ing this file requires that you have AMS-LaTeX 2.0 installed
% as well as the rest of the prerequisites for REVTeX 4.2
%
% See the REVTeX 4 README file
% It also requires running BibTeX. The commands are as follows:
%
%  1)  latex apssamp.tex
%  2)  bibtex apssamp
%  3)  latex apssamp.tex
%  4)  latex apssamp.tex
%
\documentclass[aps,prx,reprint,amsmath,amssymb,superscriptaddress,showpacs]{revtex4-1}

\usepackage{graphicx}% Include figure files
\usepackage{dcolumn}% Align table columns on decimal point
\usepackage{bm}% bold math
\usepackage{url}
%\usepackage{hyperref}% add hypertext capabilities
%\usepackage[mathlines]{lineno}% Enable numbering of text and display math
%\linenumbers\relax % Commence numbering lines

%\usepackage[showframe,%Uncomment any one of the following lines to test 
%%scale=0.7, marginratio={1:1, 2:3}, ignoreall,% default settings
%%text={7in,10in},centering,
%%margin=1.5in,
%%total={6.5in,8.75in}, top=1.2in, left=0.9in, includefoot,
%%height=10in,a5paper,hmargin={3cm,0.8in},
%]{geometry}

\begin{document}

\preprint{APS/123-QED}

\title{An Exercise in Open Data: Triple Axis Data on Si(crystal dir.)}% Force line breaks with \\


\author{Petr Cermak}
\email{cermak@mag.mff.cuni.cz}
 \affiliation{Charles University}
 
\author{J. K. Jochum}
\address{Heinz Maier-Leibnitz Zentrum (MLZ), Technische Universit\"at M\"unchen, D-85748 Garching, Germany}
\email[]{jjochum@frm2.tum.de}
\collaboration{Czech-Bavarian Mini-School on large scale facilities and open data}%\noaffiliation

\author{Charlie Author}
 \homepage{http://www.Second.institution.edu/~Charlie.Author}
\affiliation{
 Second institution and/or address\\
 This line break forced% with \\
}%
\affiliation{
 Third institution, the second for Charlie Author
}%
\author{Delta Author}
\affiliation{%
 Authors' institution and/or address\\
 This line break forced with \textbackslash\textbackslash
}%

\collaboration{CLEO Collaboration}%\noaffiliation

\date{\today}% It is always \today, today,
             %  but any date may be explicitly specified

\begin{abstract}
There will be abstract
\end{abstract}

%\keywords{Suggested keywords}%Use showkeys class option if keyword
                              %display desired
\maketitle

%\tableofcontents


\section{Introduction}

Sth on open data...

In this manuscript, we want to show how the use of digital object identifiers (DOIs) for data sets recorded at large scale facilities, can benefit researchers around the globe. 
Here, we used a data \cite{data} set made avaiable by the Institute Laue-Langevin (ILL) in Grenoble, to instruct the students of the first ``Czech-Bavarian mini-school on large scale facilities and open dat'' \cite{mini-school} on the use of open data \cite{foster}, open publishing \cite{arXiv} and the figshare platform \cite{figshare}. 

\section{Experimental details}

The data were recorded using the IN3 triple - axis spectrometer \cite{IN3} at the ILL in 2014.
We are not aware of the details of the experiments, since the experimental report, or the submitted proposal were not stored together with the data.
The sampled measured was a Silicon crystal. 
This is apparent from the sample name chosen online, and could be confirmed by calculating the lattice constant of the sample which is \textbf{XXX}, which corresponds to the \textbf{XYZ crystal direction} of Silcon.

IN3 has two different monochromators: a PG002 and a Cu monochromator.
Considering the d-value of 3.355 \AA used in the experiment, which can be extracted from the meta-data, it is clear that the PG002 monochromator was used.
In the same manner it was determined that the PG002 analyzer was used. 
Furthermore, the outgoing k-vector (k$_f$) was fixed to 2.66325 \AA$^{-1}$, which suggests the use of a PG filter. 
The outgoing k-vector corresponds to a wavelenght of $\lambda$\,=\,2.359\,\AA.

The sample was cooled down to T = 1.6\,K, where all measurements were recorded. 
This was probably done in the ``orange'' cryostat.




Please note that all numbers used here were extracted from the meta-data ONLY, and we have no way of confirming these data at the moment. 


\section{Data analysis and results}


\section{Conclusions}

Summarize results...

This data analysis would have been much easier, if together with the data, some more information regarding the experiment would have been published, e.g. an experimental report or a lab book.


\begin{thebibliography}{9}

\bibitem{ILL}
\bibitem{arXiv} arXiv.org, \url{https://arxiv.org/}
\bibitem{figshare} figshare, \url{https://figshare.com/}
\bibitem{IN3} IN3, \url{https://www.ill.eu/users/instruments/instruments-list/in3/description/instrument-layout/}
\bibitem{foster} Foster Open Data, \url{https://www.fosteropenscience.eu/}
\bibitem{mini-school} Czech-Bavarian mini-school on large scale facilities and open data \url{https://mini-school.eu/}
\bibitem{data} 
STEFFENS Paul; DELLEA Greta; DENG Yue; DIETL Christopher; DJURADO David; GAMBINO Marianna; HEPTING Matthias; INKINEN JUHO; JAFARI Atefeh; LEFRANCOIS Emilie; LOPES SELVATI Ana Carolina; PANAHI Hamed; PEDERSEN Martin Nors; PRADIP Ramu; RANIERI Umbertoluca; ROSSI Matteo; SCHATTE Sarah; STANA Markus; TIMOSENKO Janis; VONESHEN David and ZBIRI Mohamed. (2014). HSC17 Hercules practical course. Institut Laue-Langevin (ILL) doi:10.5291/ILL-DATA.TEST-2385 \url{doi:10.5291/ILL-DATA.TEST-2385}
\bibitem{example}
STEFFENS Paul; DELLEA Greta; DENG Yue; DIETL Christopher; DJURADO David; GAMBINO Marianna; HEPTING Matthias; INKINEN JUHO; JAFARI Atefeh; LEFRANCOIS Emilie; LOPES SELVATI Ana Carolina; PANAHI Hamed; PEDERSEN Martin Nors; PRADIP Ramu; RANIERI Umbertoluca; ROSSI Matteo; SCHATTE Sarah; STANA Markus; TIMOSENKO Janis; VONESHEN David and ZBIRI Mohamed. (2014). HSC17 Hercules practical course. Institut Laue-Langevin (ILL) doi:10.5291/ILL-DATA.TEST-2385

\end{thebibliography}

\end{document}
%
% ****** End of file apssamp.tex ******
