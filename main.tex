% ****** Start of file apssamp.tex ******
%
%   This file is part of the APS files in the REVTeX 4.2 distribution.
%   Version 4.2a of REVTeX, December 2014
%
%   Copyright (c) 2014 The American Physical Society.
%
%   See the REVTeX 4 README file for restrictions and more information.
%
% TeX'ing this file requires that you have AMS-LaTeX 2.0 installed
% as well as the rest of the prerequisites for REVTeX 4.2
%
% See the REVTeX 4 README file
% It also requires running BibTeX. The commands are as follows:
%
%  1)  latex apssamp.tex
%  2)  bibtex apssamp
%  3)  latex apssamp.tex
%  4)  latex apssamp.tex
%
\documentclass[%
 reprint,
%superscriptaddress,
%groupedaddress,
%unsortedaddress,
%runinaddress,
%frontmatterverbose, 
%preprint,
%preprintnumbers,
%nofootinbib,
%nobibnotes,
%bibnotes,
 amsmath,amssymb,
 aps,
%pra,
%prb,
%rmp,
%prstab,
%prstper,
%floatfix,
]{revtex4-2}

\usepackage{graphicx}% Include figure files
\usepackage{dcolumn}% Align table columns on decimal point
\usepackage{bm}% bold math
\usepackage{url}
%\usepackage{hyperref}% add hypertext capabilities
%\usepackage[mathlines]{lineno}% Enable numbering of text and display math
%\linenumbers\relax % Commence numbering lines

%\usepackage[showframe,%Uncomment any one of the following lines to test 
%%scale=0.7, marginratio={1:1, 2:3}, ignoreall,% default settings
%%text={7in,10in},centering,
%%margin=1.5in,
%%total={6.5in,8.75in}, top=1.2in, left=0.9in, includefoot,
%%height=10in,a5paper,hmargin={3cm,0.8in},
%]{geometry}

\begin{document}

\preprint{APS/123-QED}

\title{Manuscript Title:\\with Forced Linebreak}% Force line breaks with \\
\thanks{A footnote to the article title}%

\author{Petr Cermak}
\email{cermak@mag.mff.cuni.cz}
 \affiliation{Charles University}
 
\author{Second Author}%
 \email{Second.Author@institution.edu}
\affiliation{%
 Authors' institution and/or address
}%

\collaboration{Czech-Bavarian Mini-School on large scale facilities and open data}%\noaffiliation

\author{Charlie Author}
 \homepage{http://www.Second.institution.edu/~Charlie.Author}
\affiliation{
 Second institution and/or address\\
 This line break forced% with \\
}%
\affiliation{
 Third institution, the second for Charlie Author
}%
\author{Delta Author}
\affiliation{%
 Authors' institution and/or address\\
 This line break forced with \textbackslash\textbackslash
}%

\collaboration{CLEO Collaboration}%\noaffiliation

\date{\today}% It is always \today, today,
             %  but any date may be explicitly specified

\begin{abstract}
There will be abstract
\end{abstract}

%\keywords{Suggested keywords}%Use showkeys class option if keyword
                              %display desired
\maketitle

%\tableofcontents


\section{Introduction}

Sth on open data...

In this manuscript, we want to show how the use of digital object identifiers (DOIs) for data sets recorded at large scale facilities, can benefit researchers around the globe. 
Here, we used a data \cite{data} set made avaiable by the Institute Laue-Langevin in Grenoble \cite{ILL} \textbf{***cite ILL here}, to instruct the students of the first ``Czech-Bavarian mini-school on large scale facilities and open dat'' \cite{mini-school} on the use of open data, open publishing and the figshare platform \textbf{***citations for these things}. 

\section{Experimental details}

Of course we don't know, because there were not proper data description.
It was definitelly measured on IN3 and the sample was Silicon (this we know because of lettice constant XXX in the data file).

It was use PG200 as analyzer

\begin{thebibliography}{9}

\bibitem{ILL} 
\bibitem{mini-school} Czech-Bavarian mini-school on large scale facilities and open data \url{https://mini-school.eu/}
\bibitem{data} Data recorded at IN3, doi:10.5291/ILL-DATA.TEST-2385, \url{doi:10.5291/ILL-DATA.TEST-2385}
\bibitem{example}
STEFFENS Paul; DELLEA Greta; DENG Yue; DIETL Christopher; DJURADO David; GAMBINO Marianna; HEPTING Matthias; INKINEN JUHO; JAFARI Atefeh; LEFRANCOIS Emilie; LOPES SELVATI Ana Carolina; PANAHI Hamed; PEDERSEN Martin Nors; PRADIP Ramu; RANIERI Umbertoluca; ROSSI Matteo; SCHATTE Sarah; STANA Markus; TIMOSENKO Janis; VONESHEN David and ZBIRI Mohamed. (2014). HSC17 Hercules practical course. Institut Laue-Langevin (ILL) doi:10.5291/ILL-DATA.TEST-2385

\end{thebibliography}

\end{document}
%
% ****** End of file apssamp.tex ******
